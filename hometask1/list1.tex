\documentclass{article}
\usepackage{amsmath, amsthm, amssymb, amsfonts}
\usepackage{thmtools}
\usepackage{graphicx}
\usepackage{indentfirst}
\usepackage{setspace}
\usepackage{geometry}
\usepackage{float}
\usepackage{hyperref}
\usepackage{cancel}
\usepackage[utf8]{inputenc}
\usepackage[russian]{babel}
\usepackage{framed}

\newcommand{\HRule}[1]{\rule{\linewidth}{#1}}

% Стиль для решений
\declaretheoremstyle[name=Решение,]{solsty}
\declaretheorem[style=solsty,numberlike=theorem]{solution}

\setstretch{1.2}
\geometry{
    textheight=9in,
    textwidth=5.5in,
    top=1in,
    headheight=12pt,
    headsep=25pt,
    footskip=30pt
}

\title{Ниценко Снежана Андреевна 25.Б82-мм}
\author{д/з 09-10-25}
\date{}

\begin{document}
\maketitle


\begin{solution}[1.1]
\begin{enumerate}
    \item[а)] 
    
    Для $A \subseteq B \cap C$ $\Rightarrow$ $A \subseteq B$ и $A \subseteq C$. Тогда $\forall$ x $\in A$ и $x \in B \cap C$, значит $x \in B$ и $x \in C$. Следовательно, $A \subseteq B$ и $A \subseteq C$.
    
    Для $A \subseteq B$ и $A \subseteq C$ $\Rightarrow$ $A \subseteq B \cap C$. Тогда $\forall$ $x \in A$ имеем $x \in B$ и $x \in C$, значит $x \in B \cap C$. Следовательно, $A \subseteq B \cap C$.
    
    \item[б)] 
    
    Для $A \subseteq B \setminus C$ $\Rightarrow$ $A \subseteq B$ и $A \cap C = \varnothing$. Тогда $\forall$ $x \in A$ имеем $x \in B$ и $x \notin C$. Значит $A \subseteq B$ и $A \cap C = \varnothing$.
    
    Для $A \subseteq B$ и $A \cap C = \varnothing$ $\Rightarrow$ $A \subseteq B \setminus C$. Тогда $\forall$ $x \in A$ имеем $x \in B$ и $x \notin C$, значит $x \in B \setminus C$. Следовательно, $A \subseteq B \setminus C$.
\end{enumerate}
\end{solution}

\begin{solution}[1.2]
\begin{enumerate}
    \item[а)] 
    
    $\rbrack$ (пусть) $X \in \mathcal{P}(A \cap B)$. Тогда $X \subseteq A \cap B$, значит $X \subseteq A$ и $X \subseteq B$, следовательно $X \in \mathcal{P}(A)$ и $X \in \mathcal{P}(B)$, то есть $X \in \mathcal{P}(A) \cap \mathcal{P}(B)$.
    
    Обратно, $\rbrack$ $X \in \mathcal{P}(A) \cap \mathcal{P}(B)$. Тогда $X \in \mathcal{P}(A)$ и $X \in \mathcal{P}(B)$, значит $X \subseteq A$ и $X \subseteq B$, следовательно $X \subseteq A \cap B$, т.е. $X \in \mathcal{P}(A \cap B)$.
    
    \item[б)] 
    
    $\rbrack$ $X \in \mathcal{P}(A) \cup \mathcal{P}(B)$. Тогда $X \subseteq A$ или $X \subseteq B$, значит $X \subseteq A \cup B$, следовательно $X \in \mathcal{P}(A \cup B)$.
    
    \text {Пример строгого включения:} $\rbrack$ $A = \{a\}$, $B = \{b\}$. Тогда $\mathcal{P}(A) \cup \mathcal{P}(B) = \{\varnothing, \{a\}, \{b\}\}$, а $\mathcal{P}(A \cup B) = \{\varnothing, \{a\}, \{b\}, \{a,b\}\}$.
    
    \item[в)] 
    
    $\rbrack$ $X \in \mathcal{P}(A \setminus B)$. Тогда $X \subseteq A \setminus B$, значит $X \subseteq A$ и $X \cap B = \varnothing$. 
    
    Если $X = \varnothing$, то $X \in \{\varnothing\} \subseteq (\mathcal{P}(A) \setminus \mathcal{P}(B)) \cup \{\varnothing\}$.
    
    Если $X \neq \varnothing$, то $X \subseteq A$, но $X \not\subseteq B$ (так как $X \cap B = \varnothing$ и $X \neq \varnothing$), значит $X \in \mathcal{P}(A) \setminus \mathcal{P}(B)$.
    
    \text {Пример строгого включения:} $\rbrack$ $A = \{a,b\}$, $B = \{b\}$. Тогда $\mathcal{P}(A \setminus B) = \mathcal{P}(\{a\}) = \{\varnothing, \{a\}\}$, а $(\mathcal{P}(A) \setminus \mathcal{P}(B)) \cup \{\varnothing\} = (\{\varnothing, \{a\}, \{b\},\{a,b\}\} \setminus \{\varnothing, \{b\}\}) \cup \{\varnothing\} = \{\{a\}, \{a,b\}, \varnothing\}$.
\end{enumerate}
\end{solution}


\begin{solution}[1.4]
\begin{enumerate}
    \item[(а)] $f(x) = 3x + 1$: О.з. $\mathbb{R}$, инъективна (макс. 1 прообраз для каждого образа), сюръективна ( $f(x)$=y разрешимо при любом x) $\Rightarrow$ биективна.
    \item[(б)] $f(x) = x^2 + 1$: О.з. $[1, +\infty)$, не инъективна ($f(x)=f(-x)$), не сюръективна (не можем достичь отриц. значений).
    \item[(в)] $f(x) = x^3 - 1$: О.з. $\mathbb{R}$, инъективна (сущ. прообраз для каждого образа), сюръективна ($f(x)$=y всегда разрешимо) $\Rightarrow$ биективна.
    \item[(г)] $f(x) = e^x$: О.з. $(0, +\infty)$, инъективна (прообраз для каждого образа), не сюръективна (не можем достичь отриц. значений).
    \item[(д)] $f(x) = \sqrt{3x^2 + 1}$: О.з. $[1, +\infty)$, не инъективна ($f(x)=f(-x)$), не сюръективна (не можем получить отриц. значения).
    \item[(е)] $f(x) = \sin x$ на $[-\pi/2, \pi/2]$: О.з. $[-1, 1]$, инъективна (строго возрастает на заданном промежутке), не сюръективна (не получаем значения меньше -1 и больше 1).
    \item[(ж)] $f(x) = \sin x$ на $[0, \pi]$: О.з. $[0, 1]$, не инъективна ($f(\pi/3)=f(2\pi/3)$), не сюръективна.
    \item[(з)] $f(x) = \sin x$ на $\mathbb{R} \to [-1,1]$: О.з. $[-1,1]$, не инъективна (sin периодичен), сюръективна (все зн. [-1,1] достигаются).
    \item[(и)] $f(x) = x^2 \sin x$: О.з. $\mathbb{R}$, не инъективна ($f(0)=f(\pi)$), сюръективна (для каждого y сущ. x).
\end{enumerate}
\end{solution}

\begin{solution}$[1.5]$ $g\circ f$ = $f(g(x))$
\begin{enumerate}
    \item[(а)] {Неверно}. $f(x) = x$, $g(x) = x^2$. $g\circ f$ = $x^2$ - не инъективна.
    
    \item[(б)] {Верно}. $g(x) = x$, $f(x) = x$. $g\circ f$ = x - сюръективна.
    
    \item[(в)] {Верно}. $f(x) = x$, $g(x) = x+1$. $g\circ f$ = x+1 - биективна.
    
    \item[(г)] Неверно. Если $A = \{x\}$, $B = \{y,z\}$, $C = \{u\}$, $g(x) = y$ $\Rightarrow$ $f(y) = u$, $f(z) = u$.
    
    \item[(д)] Верно. $\rbrack$ $g(x_1) = g(x_2)$. $\Rightarrow$ $f(g(x_1)) = f(g(x_2))$, значит $(g\circ f)(x_1) = (g\circ f)(x_2)$. Так как $g\circ f$ инъективна, $x_1 = x_2$.

    
    \item[(е)] {Неверно}. $A = \{x,y\}$, $B = \{z,w\}$, $C = \{u\}$, $g(x)=z$, $g(y)=w$, $f(z)=u$, $f(w)=u$. $\Rightarrow$ $g\circ f$ сюръективна, но $f$ не сюръективна.
\end{enumerate}
\end{solution}

\end{document}
