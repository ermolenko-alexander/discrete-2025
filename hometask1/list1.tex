\documentclass[12pt]{article}
\usepackage[margin=2.2cm]{geometry}
\usepackage[russian]{babel}
\usepackage{amsmath, amssymb, amsthm}
\usepackage{graphicx}
\usepackage{hyperref}

\begin{document}
\maketitle 
\title Орел Владислав Олегович Б82-ММ
\item1.1
а) Доказать: \(A \subseteq B \cap C \Leftrightarrow A \subseteq B \ \text{и}\ A \subseteq C\).

Док-во.
\begin{itemize}
\item \((\Rightarrow)\) Пусть \(x \in A\). следует \(A \subseteq B \cap C\) следует \(x \in B \cap C\), то есть \(x \in B\) и \(x \in C\). Следовательно, \(A \subseteq B\) и \(A \subseteq C\).
\item \((\Leftarrow)\) Если \(A \subseteq B\) и \(A \subseteq C\), то для любого \(x \in A\) имеем \(x \in B\) и \(x \in C\), значит \(x \in B \cap C\). Следовательно, \(A \subseteq B \cap C\).
\end{itemize}
\item б) Доказать: A \subseteq B\backslash C\Leftrightarrow A \subseteq B \ \text{и} \ A\cap C=\O.

Док-во.
\begin{itemize}
\item \((\Rightarrow)\) Пусть \(x \in A\) => \(x \in B), \ \text{т.к} \ A \subseteq B. A \cap C=\O => x \notin C.
=> по опр. x \in B\backslash C => A \subseteq B\backslash C
\item \((\Leftarrow)\) Пусть \(x \in A\) => x \in B \ \text{и} \ x \notin C, => A \subseteq B \ \text{и} \ x \notin A\cap C =>A \subseteq B и A\cap C=\O.
=> по опр. x \in B\backslash C => A \subseteq B\backslash C
\end{itemize}

\item 1.2 
а) Доказать равенство: \(\mathcal{P}(A \cap B) = \mathcal{P}(A) \cap \mathcal{P}(B)\).

Док-во.
\begin{itemize}
\item \((\subseteq)\) Пусть \(X \in \mathcal{P}(A \cap B)\). Тогда \(X \subseteq A \cap B\), откуда \(X \subseteq A\) и \(X \subseteq B\). Следовательно, \(X \in \mathcal{P}(A)\) и \(X \in \mathcal{P}(B)\), то есть \(X \in \mathcal{P}(A) \cap \mathcal{P}(B)\).
\item \((\supseteq)\) Пусть \(X \in \mathcal{P}(A) \cap \mathcal{P}(B)\). Тогда \(X \subseteq A\) и \(X \subseteq B\). Следовательно, \(X \subseteq A \cap B\), то есть \(X \in \mathcal{P}(A \cap B)\).
\end{itemize}

\item б) Доказать: \(\mathcal{P}(A) \cup \mathcal{P}(B) \subseteq \mathcal{P}(A \cup B)\).

Док-во: Пусть \(X \in \mathcal{P}(A) \cup \mathcal{P}(B)\). Тогда \(X \subseteq A\) или \(X \subseteq B\). В любом случае \(X \subseteq A \cup B\), следовательно, \(X \in \mathcal{P}(A \cup B)\).

Пример: A=\{1\}, B=\{1,2\}

=>\mathcal{P}(A \cup B)=\{\O,\{1\},\{2\},\{1,2\}\}

\mathcal{P}(A)=\{\O,1\}, \mathcal{P}(B)=\{\O,2\} =>\mathcal{P}(A)\cup \mathcal{P}(B)=\{\O,\{1\},\{2\}\}

=>\mathcal{P}(A)\cup \mathcal{P}(B) \subset \mathcal{P}(A \cup 

\item в) Доказать: \mathcal{P}(A \backslash B) \subseteq (\mathcal{P}(A) \backslash \mathcal{P}(B)) \cup \{\O\}.

Док-во: 

1) Пусть X \in \mathcal{P}(A \backslash B) => X \subseteq A \backslash B => X \subseteq A \ \text{и} \ X \nsubseteq B \ 
\text{Если} \ X=\O, => X \subseteq A \ \text{и} \ X \subseteq B \ 
\text{Если} \ X\neq\O, => X \subseteq A \ \text{и} \ X \cap B=\O => X \in \mathcal{P}(A) \backslash \mathcal{P}(B) \cup \{\O\}

Пример: A=\{1,2\}, B=\{2\}

=>\mathcal{P}(A \backslash B)=\{\O,\{1\}\}

\mathcal{P}(A)=\{\O,1\}, \mathcal{P}(B)=\{\O,2\} =>\mathcal{P}(A)\backslash \mathcal{P}(B)=\{\{1\},\{1,2\}\}

=>\mathcal{P}(A \backslash B) \subset \mathcal{P}(A)\backslash \mathcal{P}(B)\cup \{\O\}

\item1.3
\item a) Проверка св-в ОЧП:

    1) Рефлексивность:  \forall v \ \text{должно выполняться} \ (v,v) \in R_1
    для i_k<=i_k \ \text{будет выполняться, а для} \ i_k<i_k\ \ \text{-- нет} \
    => \ \text{нерефлексивно}

    2) Антисимметричность: условия \exists k : i_k<j_k, \ \text{и} \ i_k=j_k \ \text{не могут выполняться одновременно => неантисимметрично} 
=> R_1 - \ \text{не ОЧП} \ 


\item б) Проверка св-в ОЧП: \

    1) Рефлексивность: \exists k : i_l=i_l => i_k<i_k \ \text{-- неверно} \
    => \ \text{нерефлексивно}

    2) Антисимметричность: условия \exists k : i_l=j_l, \ \text{и} \ i_k<j_k, i_l=j_l, \ \text{и} \ i_k<j_k \ \text{не могут выполняться одновременно => неантисимметрично}

    => R_2 - \ \text{не ОЧП} \ 

\item 1.4

a) E(f)=R, биекция

б) E(f)>=1, ничего

в) E(f)=R, биекция

г) E(f)>0, инъективна

д) E(f)>=2, ничего

е) E(f)=[-1;1], биекция

ж) E(f)=[0;1], ничего

з) E(f)=[-1;1], сюръективна

и) E(f)=R, сюръективна

\item 1.5

а)неверно, т.к f может быть неинъективна

б)верно, т.к f(x)=z => g(f(x))=g(z)=y

в)верно, т.к f - инъективна и сюръективна и g - инъективна и сюръективна

г)верно, т.к g(f(x1))=g(f(x2)) => f(x1)=f(x2) => x1=x2

д)неверно, т.к f может быть неинъективна

е)верно, т.к для каждого y существует x т.ч g(f(x))=y => f(x) должен быть определен для всех y

\item 1.8

Пусть A-множество жителей, k_i-\text{число знакомых для каждого жителя}


C_1-\text{множество людей с } d_i>=0,5*|A|, \ C_2-\text{множество людей с } d_i<=0,5*|A|

=> C_2=A \backslash C_1

По условию d_i>=0,3*|A| \ => для C_1 - 0,3*|A|<=d_i<=0,5*|A|    

=> C_1 \cap C_2 \neq\O => \text{такое возможно}

\end{document}
