\documentclass{article}
\usepackage{amsmath, amsthm, amssymb, amsfonts}
\usepackage{thmtools}
\usepackage{graphicx}
\usepackage{indentfirst}
\usepackage{setspace}
\usepackage{geometry}
\usepackage{float}
\usepackage{hyperref}
\usepackage{cancel}
\usepackage[utf8]{inputenc}
\usepackage[russian]{babel}
\usepackage{framed}
\usepackage[dvipsnames]{xcolor}
\usepackage{tcolorbox}
\usepackage[textsize=small, textwidth=3.5cm]{todonotes}
\setlength{\marginparwidth}{1.5cm}

\colorlet{LightGray}{White!90!Periwinkle}
\colorlet{LightOrange}{Orange!15}
\colorlet{LightGreen}{Green!15}

\newcommand{\HRule}[1]{\rule{\linewidth}{#1}}

% Теоремные окружения и стили
\declaretheoremstyle[name=Theorem,]{thmsty}
\declaretheorem[style=thmsty,numberwithin=section]{theorem}
\tcolorboxenvironment{theorem}{colback=LightGray}

\declaretheoremstyle[name=Conjecture,]{thmsty}
\declaretheorem[style=thmsty,numberwithin=section]{conjecture}
\tcolorboxenvironment{conjecture}{colback=LightGray}

\declaretheoremstyle[name=Definition,]{thmsty}
\declaretheorem[style=thmsty,numberwithin=section]{definition}
\tcolorboxenvironment{definition}{colback=LightGray}

% Стиль задач: показываем пометку \NOTE в квадратных скобках и русское имя
\declaretheoremstyle[name=Задача,notebraces={[}{]}]{prosty}
\declaretheorem[style=prosty,numberlike=theorem]{task}

\setstretch{1.2}
\geometry{
    textheight=9in,
    textwidth=5.5in,
    top=1in,
    headheight=12pt,
    headsep=25pt,
    footskip=30pt,
    right=4.5cm
}

\setuptodonotes{size=\small, color=white, bordercolor=white, textcolor=Bittersweet}

\title{Листочек №1 ``Теория множеств''}
\author{Матмех, группы 25.Б82-мм \\ Никитин Артем Сергеевич}
\date{Октябрь 2025}

\begin{document}
\maketitle

\section{Задачи}

% 1. Множества — эквивалентные формулировки
\begin{task}[1]
Доказать, что
\begin{enumerate}
    \item[а)] $A \subseteq B \cap C \iff A \subseteq B$ и $A \subseteq C$;
    \item[б)] $A \subseteq B \setminus C \iff A \subseteq B$ и $A \cap C = \varnothing$.
\end{enumerate}
\end{task}

\begin{proof}[Решение]\
\begin{enumerate}
    \item[а)] Доказательство:
    
    $(\Rightarrow)$ Пусть $A \subseteq B \cap C$, тогда для любого $x \in A$ имеем $x \in B \cap C$, значит $x \in B$ и $x \in C$, тогда $A \subseteq B$ и $A \subseteq C$.
    
    $(\Leftarrow)$ Пусть $A \subseteq B$ и $A \subseteq C$, тогда для любого $x \in A$ имеем $x \in B$ и $x \in C$, значит $x \in B \cap C$, тогда $A \subseteq B \cap C$.
    
    \item[б)] Доказательство:
    
    $(\Rightarrow)$ Пусть $A \subseteq B \setminus C$, тогда для любого $x \in A$ имеем $x \in B$ и $x \notin C$, тогда $A \subseteq B$ и $A \cap C = \varnothing$.
    
    $(\Leftarrow)$ Пусть $A \subseteq B$ и $A \cap C = \varnothing$, тогда для любого $x \in A$ имеем $x \in B$ и $x \notin C$ \, тогда $x \in B \setminus C$, значит $A \subseteq B \setminus C$.
\end{enumerate}
\end{proof}

% 2. Мощности множеств — равенства и включения
\begin{task}[2]
Доказать следующие равенства и включения:
\begin{enumerate}
    \item[а)] $\mathcal{P}(A \cap B) = \mathcal{P}(A) \cap \mathcal{P}(B)$;
    \item[б)] $\mathcal{P}(A \cup B) \supseteq \mathcal{P}(A) \cup \mathcal{P}(B)$;
    \item[в)] $\mathcal{P}(A \setminus B) \subseteq (\mathcal{P}(A) \setminus \mathcal{P}(B)) \cup \{\varnothing\}$.
\end{enumerate}
Привести примеры, когда указанные включения являются строгими.
\end{task}

\begin{proof}[Решение]\
\begin{enumerate}
    \item[а)] Доказательство:
    
    $(\subseteq)$ Пусть $X \in \mathcal{P}(A \cap B)$, тогда $X \subseteq A \cap B \subseteq A$ и $X \subseteq A \cap B \subseteq B$, значит $X \in \mathcal{P}(A)$ и $X \in \mathcal{P}(B)$, тогда $X \in \mathcal{P}(A) \cap \mathcal{P}(B)$.
    
    $(\supseteq)$ Пусть $X \in \mathcal{P}(A) \cap \mathcal{P}(B)$, тогда $X \subseteq A$ и $X \subseteq B$, значит $X \subseteq A \cap B$, следовательно $X \in \mathcal{P}(A \cap B)$.
    
    \item[б)] Доказательство:
    
    Пусть $X \in \mathcal{P}(A) \cup \mathcal{P}(B)$, тогда $X \subseteq A$ или $X \subseteq B$, значит $X \subseteq A \cup B$, тогда $X \in \mathcal{P}(A \cup B)$.
    
     Пример: $A = \{0\}$, $B = \{1\}$. Тогда $\{0,1\} \in \mathcal{P}(A \cup B)$, но $\{0,1\} \notin \mathcal{P}(A) \cup \mathcal{P}(B)$, поэтому включение строгое.
    
    \item[в)] Доказательство:
    
    Пусть $X \in \mathcal{P}(A \setminus B)$, тогда $X \subseteq A \setminus B \subseteq A$, поэтому $X \in \mathcal{P}(A)$,сли $X \neq \varnothing$, то $X$ содержит хотя бы один элемент из $A \setminus B$, значит $X \not\subseteq B$, поэтому $X \notin \mathcal{P}(B)$, тогда $X \in \mathcal{P}(A) \setminus \mathcal{P}(B)$.
    
    Пример: $A = \{0,1\}$, $B = \{0\}$. Тогда $\mathcal{P}(A \setminus B) = \mathcal{P}(\{1\}) = \{\varnothing, \{1\}\}$, а $(\mathcal{P}(A) \setminus \mathcal{P}(B)) \cup \{\varnothing\} = \{\varnothing, \{1\}, \{0,1\}\}$.
\end{enumerate}
\end{proof}

% 3. Алфавиты и порядки
\begin{task}[3]
Пусть $A = \{a_1, \ldots, a_m\}$ — конечный алфавит, $A^n$ — множество слов длины $n$ в алфавите $A$.
\begin{enumerate}
    \item[(a)] На $A^n$ задано отношение $R_1$: для $v = a_{i_1}\ldots a_{i_n}$ и $w = a_{j_1}\ldots a_{j_n}$ положим $(v,w)\in R_1 \iff i_k\le j_k$ для всех $k=1,\dots,n$ и $i_k<j_k$ для некоторого $k$. Является ли $R_1$ отношением частичного (линейного) порядка?
    \item[(б)] На $A^*$ задано отношение $R_2$: для $v = a_{i_1}\ldots a_{i_n}$ и $w = a_{j_1}\ldots a_{j_r}$ положим $(v,w)\in R_2 \iff \exists k$ от $1$ до $n$ с $i_\ell=j_\ell$ при $1\le \ell<k$ и $i_k<j_k$, причем первые $n$ символов $w$ совпадают со словом $v$. Является ли $R_2$ отношением частичного (линейного) порядка?
\end{enumerate}
\end{task}

\begin{proof}[Решение]\
\begin{enumerate}
    \item[(a)] Нет, не является, так как отсутствует рефлексивность (требуется $i_k < j_k$ для некоторого $k$) и антисимметричность ($i_k \le j_k$ и $j_k \le i_k$ для всех $k$, значит $i_k = j_k$ для всех $k$, противоречие с условием $i_k < j_k$ для некоторого $k$).
    
    
    \item[(б)] Нет, не является. По условию, первые $n$ символов $w$ совпадают с первыми $n$ символами $v$, но длина $v$ равна $n$, поэтому слово $w$ можно представить как слово $v$ с последующими $a_{j_k}$, где $k > n$, но это противоречие условию $\exists k$ от $1$ до $n$ с $i_\ell=j_\ell$ при $1\le \ell<k$ и $i_k<j_k$.
\end{enumerate}
\end{proof}

% 4. Функции — область значений и свойства
\begin{task}[2]
Для каждой из функций найти область значений и указать, является ли функция инъективной, сюръективной, биекцией.
\begin{enumerate}
    \item[(а)] $f : \mathbb{R} \to \mathbb{R},\ f(x) = 3x + 1$;
    \item[(б)] $f : \mathbb{R} \to \mathbb{R},\ f(x) = x^2 + 1$;
    \item[(в)] $f : \mathbb{R} \to \mathbb{R},\ f(x) = x^3 - 1$;
    \item[(г)] $f : \mathbb{R} \to \mathbb{R},\ f(x) = e^x$;
    \item[(д)] $f : \mathbb{R} \to \mathbb{R},\ f(x) = \sqrt{3x^2 + 1}$;
    \item[(е)] $f : [-\pi/2, \pi/2] \to \mathbb{R},\ f(x) = \sin x$;
    \item[(ж)] $f : [0, \pi] \to \mathbb{R},\ f(x) = \sin x$;
    \item[(з)] $f : \mathbb{R} \to [-1, 1],\ f(x) = \sin x$;
    \item[(и)] $f : \mathbb{R} \to \mathbb{R},\ f(x) = x^2 \sin x$.
\end{enumerate}
\end{task}

\begin{proof}[Решение]\
\begin{enumerate}
    \item[(а)] Область значений: $\mathbb{R}$ \\
     Свойства: инъективна, сюръективна, биекция
    
    \item[(б)] Область значений: $[1, +\infty)$ \\
     Свойства: не инъективна, не сюръективна 
    
    \item[(в)] Область значений: $\mathbb{R}$ \\
     Свойства: инъективна, сюръективна, биекция
    
    \item[(г)] Область значений: $(0, +\infty)$ \\
     Свойства: инъективна, не сюръективна
    
    \item[(д)] Область значений: $[1, +\infty)$ \\
    Свойства: не инъективна, не сюръективна
    
    \item[(е)] Область значений: $[-1, 1]$ \\
    Свойства: инъективна, не сюръективна
    
    \item[(ж)] Область значений: $[0, 1]$ \\
    Свойства: не инъективна, не сюръективна
    
    \item[(з)] Область значений: $[-1, 1]$ \\
    Свойства: не инъективна, сюръективна
    
    \item[(и)] Область значений: $\mathbb{R}$ \\
    Свойства: не инъективна, сюръективна
\end{enumerate}
\end{proof}

% 5. Композиция функций — логические следствия
\begin{task}[2]
Даны $g : A \to B$ и $f : B \to C$. Рассмотрим композицию $g\circ f : A \to C$, $(g\circ f)(x)=f(g(x))$. Определить, какие утверждения верны:
\begin{enumerate}
    \item[(а)] Если $g$ инъективна, то $g\circ f$ инъективна.
    \item[(б)] Если $f$ и $g$ сюръективны, то $g\circ f$ сюръективна.
    \item[(в)] Если $f$ и $g$ биекции, то $g\circ f$ биекция.
    \item[(г)] Если $g\circ f$ инъективна, то $f$ инъективна.
    \item[(д)] Если $g\circ f$ инъективна, то $g$ инъективна.
    \item[(е)] Если $g\circ f$ сюръективна, то $f$ сюръективна.
\end{enumerate}
\end{task}

\begin{proof}[Решение]\
\begin{enumerate}
    \item[(а)] Неверно. Например, если $A = \{a\}$, $B = \{b,c\}$, $C = \{d\}$, $g(a) = b$, тогда $f(b) = d$, $f(c) = d$.
    
    \item[(б)] Верно. Пусть $c \in C$. Так как $f$ сюръективна, существует $b \in B$ такой что $f(b) = c$, так как $g$ сюръективна, существует $a \in A$ такой что $g(a) = b$, тогда $(g\circ f)(a) = f(g(a)) = f(b) = c$.
    
    \item[(в)] Верно. Из (б) и того, что композиция инъективных функций инъективна.
    
    \item[(г)] Неверно. Например, если $A = \{a\}$, $B = \{b,c\}$, $C = \{d\}$, $g(a) = b$, тогда $f(b) = d$, $f(c) = d$.
    
    \item[(д)] Верно. Пусть $g(a_1) = g(a_2)$. Тогда $f(g(a_1)) = f(g(a_2))$, значит $(g\circ f)(a_1) = (g\circ f)(a_2)$. Так как $g\circ f$ инъективна, $a_1 = a_2$.
    
    \item[(е)] Верно. Пусть $g(a) = b$, $f(b) = c$, тогда $(g\circ f)(a) = f(g(a)) = f(b) = c $ - сюръективна.
\end{enumerate}
\end{proof}

% 6. Кафе-мороженое (парадокс ссор)
\begin{task}[3]
Учащиеся одной школы часто собираются группами и ходят в кафе-мороженое. После такого посещения они ссорятся настолько, что никакие двое из них после этого вместе мороженое не едят. К концу года выяснилось, что в дальнейшем они могут ходить в кафе-мороженое только поодиночке. Докажите, что если число посещений было к этому времени больше 1, то оно не меньше числа учащихся в школе.
\end{task}

\begin{proof}[Решение]
.
\end{proof}

% 7. Вечерние визиты класса
\begin{task}[3]
30 учеников одного класса решили побывать друг у друга в гостях. Известно, что ученик за вечер может сделать несколько посещений, и что в тот вечер, когда к нему кто-нибудь должен прийти, он сам никуда не уходит. Покажите, что для того, чтобы все побывали в гостях у всех,
\begin{enumerate}
    \item[а)] четырех вечеров недостаточно,
    \item[б)] пяти вечеров также недостаточно,
    \item[в)] а десяти вечеров достаточно,
    \item[г)] и даже семи вечеров тоже достаточно.
\end{enumerate}
\end{task}

\begin{proof}[Решение]\
Пусть $S$ — множество вечеров, в которые происходят посещения. Каждому ученику сопоставим подмножество $A \subset S$ — множество вечеров, когда он ходит в гости. Два ученика с множествами $A$ и $B$ могут взаимно посетить друг друга тогда и только тогда, когда $A \not\subset B$ и $B \not\subset A$. Таким образом, чтобы все 30 учеников попарно могли встретиться, нужно выбрать 30 подмножеств $S$, никакие два из которых не находятся в отношении включения (то есть будем искать антицепи).

Когда $|S| \le 5$ максимальная антицепь имеет размер $\binom{5}{2} = 10$ (или $\binom{5}{3} = 10$). Так как $10 < 30$, снова нельзя выбрать 30 подмножеств без включения. Следовательно, пяти вечеров и меньше не достаточно.

Когда $|S| \ge 7$, возьмем все подмножества размера 3: их не менее $\binom{7}{3} = 35$. Ни одно трехэлементное подмножество не содержится в другом трехэлементном. Так как $35 \ge 30$, можно выбрать 30 подмножеств размера 3, и все ученики смогут взаимно посетить друг друга. Следовательно, семи вечеров и больше достаточно.
\end{proof}

% 8. Выборы мэра и знакомые
\begin{task}[3]
У каждого из жителей города $N$ число знакомых составляет не менее 30\% населения города. Житель идет на выборы, если баллотируется хотя бы один из его знакомых. Докажите, что можно так провести выборы мэра города $N$ из двух кандидатов, что в них примет участие не менее половины жителей.
\end{task}

\begin{proof}[Решение]\

\end{proof}

% 9. Комитеты в Думе
\begin{task}[3]
В Думе 1600 депутатов образовали 16000 комитетов по 80 человек в каждом. Докажите, что найдутся два комитета, имеющие не менее четырех общих членов.
\end{task}

\begin{proof}[Решение]\
Предположим, что у любых двух комитетов $\le 3$ общих членов. Всего неупорядоченных пар комитетов $\frac{16000 \times 15999}{2} = 127\,992\,000$. Если у каждой пары комитетов $\le 3$ общих членов, то общее число ситуаций «депутат $X$ сидит в комитетах $A$ и $B$ одновременно» для всех пар комитетов не больше $3 \times 127\,992\,000 = 383\,976\,000$.

Теперь подсчитаем это же число другим способом. Всего мест в комитетах: $16000 \times 80 = 1\,280\,000$. Среднее число комитетов на депутата $ \frac{1\,280\,000}{1600} = 800$. Для депутата, который сидит в $a$ комитетах, он является общим членом для $\frac{a(a-1)}{2}$ пар комитетов. 

Найдем минимальное общее число ситуаций «депутат $X$ сидит в комитетах $A$ и $B$ одновременно». Это число будет минимальным, когда количество комитетов на каждого депутата будет равно среднему числу комитетов. Если все депутаты сидят ровно в $a = 800$ комитетах, то каждый депутат дает $\frac{800 \times 799}{2} = 400 \times 799 = 319\,600$ пар. Тогда найдем минимальное значение «депутат $X$ сидит в комитетах $A$ и $B$ одновременно»: $1600 \times 319\,600 = 511\,360\,000$.

Получаем противоречие: $511\,360\,000 > 383\,976\,000$ (минимальное больше максимального). Значит, предположение неверно, и существуют два комитета, имеющие не менее четырех общих членов. 
\end{proof}

\vspace{1.5em}
\noindent\textbf{Примечание.}
\begin{quote}
Напоминание: задачи, имеющие сложность 1 должны уметь решать все. На решение этих задач дается дедлайн – две недели (на первый раз 09.10.2025).
\end{quote}

\end{document}
